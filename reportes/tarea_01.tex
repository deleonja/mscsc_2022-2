\documentclass[11pt,letterpaper]{article}
\usepackage[utf8]{inputenc}
\usepackage[spanish]{babel}
\usepackage{amsmath}
\usepackage{amsfonts}
\usepackage{amssymb}
\usepackage{physics}
\usepackage{hyperref}
\usepackage[left=2cm,right=2cm,top=2cm,bottom=2cm]{geometry}

\usepackage[draft,inline,nomargin]{fixme} \fxsetup{theme=color}
\definecolor{jacolor}{RGB}{200,40,0} \FXRegisterAuthor{ja}{aja}{\color{jacolor}JA}

\newcommand{\mcM}{\mathcal{M}}

\renewcommand{\labelenumii}{\arabic{enumi}.\arabic{enumii}}
\renewcommand{\labelenumiii}{\arabic{enumi}.\arabic{enumii}.\arabic{enumiii}}
\renewcommand{\labelenumiv}{\arabic{enumi}.\arabic{enumii}.\arabic{enumiii}.\arabic{enumiv}}

%%%%% Author
\author{José Alfredo de León}

%%%% Title 
\title{Tarea 1\\
\large{Métodos de Simulación Computacional para Sistemas Cuánticos - 2022-2}}


\begin{document}
\date{15 de febrero de 2022}
\maketitle

\section{Objetivos}
\subsection{Objetivo general}
Implementar desde cero el método de eliminación de Gauss-Jordan en Mathematica.

\subsection{Objetivos específicos}
\begin{enumerate}
\item Utilizar el paradigma de la programación modular.
\item Implementar un código cuyas rutinas puedan ser reutilizadas
para otras implementaciones.
\item Tomar ventaja del paradigma de Mathematica.
\end{enumerate}

\section{El problema}
La resolución de los sistemas de ecuaciones lineales, dependiendo de su tamaño,
puede convertirse en una tarea titánica e impráctica de realizar a mano. 
Por esta razón, una solución práctica es implementar numéricamente
la resolución de los sistemas de ecuaciones lineales. 

Para este trabajo vamos a limitarnos a un método en particular: eliminación de
Gauss-Jordan. Para ilustrar qué hace el método consideremos el sistema de ecuaciones
\begin{align}\label{eq:linear_system}
a_{i,1}x_1+\ldots+a_{i,j}x_j+\ldots+a_{i,n}x_n=b_{i}, 
\quad a_{i,j}\in \mathbb{R},\quad
i=1,2,\ldots,k,\quad j=1,2,\ldots,n,
\end{align}
donde $k$ puede ser como máximo igual a $n$.
Del álgebra lineal se sabe que para $n$ variables $x_j$ se tienen hasta $n$
ecuaciones linealmente independientes. Razón por la cual, sin pérdida de la
generalidad, cualquier sistema lineal con más ecuaciones que variables puede
reducirse a \eqref{eq:linear_system}. El sistema \eqref{eq:linear_system}
puede escribirse en forma matricial como 
\begin{align}
\mqty(a_{1,1}&\dots&a_{1,j}&\dots&a_{1,n}\\
\vdots&\ddots&\vdots&\ddots&\vdots\\
a_{i,1}&\dots&a_{i,j}&\dots&a_{i,n}\\
\vdots&\ddots&\vdots&\ddots&\vdots\\
a_{n,1}&\dots&a_{n,j}&\dots&a_{n,n})
\mqty(x_1\\ \vdots\\ x_i\\ \vdots\\ x_n)=
\mqty(b_1\\ \vdots\\ b_i\\ \vdots\\ b_n),
\end{align}
cuya solución se reduce a transformar la matriz aumentada
\begin{align}\label{eq:augmented_matrix}
\mqty(A&\vec b)=\mqty(a_{1,1}&\dots&a_{1,j}&\dots&a_{1,n}&b_1\\
\vdots&\ddots&\vdots&\ddots&\vdots&\vdots\\
a_{i,1}&\dots&a_{i,j}&\dots&a_{i,n}&b_i\\
\vdots&\ddots&\vdots&\ddots&\vdots&\vdots\\
a_{n,1}&\dots&a_{n,j}&\dots&a_{n,n}&b_n)
\end{align}
hasta la forma
\begin{align}\label{eq:lin_eq_solution}
\mqty(\mathbb{I}_n&\vec c)=
\mqty(1&\dots&0&\dots&0&c_1\\
\vdots&\ddots&\vdots&\ddots&\vdots&\vdots\\
0&\dots&1&\dots&0&c_i\\
\vdots&\ddots&\vdots&\ddots&\vdots&\vdots\\
0&\dots&0&\dots&1&c_n),
\end{align}
en la que hay una matriz idendidad de orden $n$ en las antiguas posiciones 
de la submatriz $A$ y la última columna con los elementos $c_i$ es lo
que resulta de aplicar las operaciones correspondientes a las filas de 
\eqref{eq:augmented_matrix} para transformarla a la forma \eqref{eq:lin_eq_solution}.
La solución del sistema de ecuaciones \eqref{eq:linear_system} es
$x_i=c_i$.

Debemos de resaltar que los sistemas de la forma 
\eqref{eq:linear_system} con infinitas soluciones no pueden transformarse hasta la 
forma \eqref{eq:lin_eq_solution}, sino únicamente hasta la forma de una
matriz triangular superior en las antiguas posiciones de la submatriz $A$.

El método de Gauss-Jordan consiste entonces de dos partes:
\begin{enumerate}
\item Fase de avance: en esta fase se hacen las operaciones necesarias
entre las filas de la matriz \eqref{eq:augmented_matrix} de tal manera que
se lleve hasta la forma de una matriz triangular superior. En esta fase el objetivo
es transformar la matriz \eqref{eq:augmented_matrix} hasta una matriz 
de la forma 
\begin{align}\label{eq:upper_triang}
\mqty(1&\dots&\tilde{a}_{1,j}&\dots&\tilde{a}_{1,n}&c_1\\
\vdots&\ddots&\vdots&\ddots&\vdots&\vdots\\
0&\dots&1&\dots&\tilde{a}_{i,j}&c_i\\
\vdots&\ddots&\vdots&\ddots&\vdots&\vdots\\
0&\dots&0&\dots&1&c_n),
\end{align}
\item Fase de retroceso: en esta fase se hacen las operaciones necesarias
entre las filas de la matriz triangular superior que resultó de la fase de avance
para conseguir una matriz de la forma \eqref{eq:lin_eq_solution}. 
\end{enumerate}
Es necesario mencionar que para los sistemas de ecuaciones con infinitas 
soluciones no se puede transformar la matriz \eqref{eq:augmented_matrix}
hasta la forma de la matriz \eqref{eq:upper_triang}, sino hasta una matriz
que es triangular superior, pero con uno o más ceros sobre la diagonal. 

En la implementación numérica pueden ocurrir errores de redondeo. 
Por esto, una forma para evitar este tipo de problemas se utiliza
el pivoteo parcial o total. En general, el pivoteo consiste en intercambiar
filas y columnas para mover el elemento mayor en valor absoluto de la 
submatriz $A$ hacia la posición de $a_{k,k}$ antes de hacer las operaciones
sobre las filas necesarias para hacer cero los elementos de matriz por
debajo de la posición de  $a_{k,k}$. En el pivoteo parcial 
se intercambian sólo filas y el objetivo es mover el número mayor en 
valor absoluto en la $k$-ésima columna. En el pivoteo 
total se intercambian filas y columnas para llevar el elemento mayor en valor absoluto,
en la submatriz de $A$ formada borrando las filas de arriba y columnas a la
izquierda de las $k$-ésimas fila y columna, hacia la posición de $a_{k,k}$.

\section{Parte 1: Ejercicios a mano}
Refiérase al repositorio de github \href{https://github.com/deleonja/mscsc\_2022-2}
{https://github.com/deleonja/mscsc\_2022-2}, al archivo \verb|tarea_01.nb|
en la dirección \verb|codigos/|, para encontrar el notebook de Mathematica
en el que se hicieron los cálculos a mano en la sección ``Parte 1: Ejercicios a mano''. 

\section{Parte 2: Implementación Gauss-Jordan}
\subsection{Algoritmo}
El algoritmo detrás de nuestra implementación de Gauss-Jordan
es el que enunciamos a continuación.

Entrada: matriz aumentada $\mcM$ de la forma \eqref{eq:augmented_matrix} 
de un sistema de ecuaciones como en \eqref{eq:linear_system}. Salida:
lista con la matriz escalonada de $\mcM$ y vector con el registro de
los intercambios de las columnas durante el pivoteo.
\begin{enumerate}
\item Para cada fila $i$ de $\mcM$:
\begin{enumerate}
\item Si la submatriz de $A$ 
[vea la ec. \eqref{eq:augmented_matrix}] que se forma borrando las filas
y columnas arriba y hacia la izquierda de la posición de $\mcM_{i,i}$ es distinta
de la matriz \textbf{0}, entonces:
\begin{enumerate}
\item Buscar el elemento pivotante $p$ en la submatriz de $A$ 
[vea la ec. \eqref{eq:augmented_matrix}] que se forma borrando las filas
y columnas arriba y hacia la izquierda de la posición de $\mcM_{i,i}$. La posición 
del elemento pivotante es $(p_i,p_j)$.
\item Guardar registro del intercambio de columnas.
\item Intercambiar las filas $\{i,p_i\}$ y las columnas $\{j,p_j\}$ en $\mcM$, y multiplicar
la nueva fila $i$ de $\mcM$ por $1/p$. De esta manera conseguimos hacer los intercambios
y operaciones necesarias para colocar un 1 en la posición $(i,i)$ de $\mcM$.
\item Hacer las operaciones entre filas necesarias para hacer cero todos 
los elementos de matriz $\mcM_{k,i}$, para $k=i+1,i+2,\ldots,n$.
\end{enumerate}
\end{enumerate}
\item Para cada fila $i$ que tiene un 1 en la posición $(i,i)$ de la $\mcM$ 
transormada en el paso 1.1:
\begin{enumerate}
\item Hacer las operaciones entre filas necesarias para hacer cero todos 
los elementos de matriz $\mcM_{k,i}$, para $k=1,2,\ldots,i-1$.
\end{enumerate}
\end{enumerate}

\subsection{Implementación}
Refiérase al repositorio de github \href{https://github.com/deleonja/mscsc\_2022-2}
{https://github.com/deleonja/mscsc\_2022-2}, al archivo \verb|tarea_01.nb|
en la dirección \verb|codigos/|, para encontrar el notebook de Mathematica
en el que se detalla la implementación en la sección ``Parte 2: Implementación
Gauss-Jordan''. 
Para la implementación utilizamos
el paradigma de programación modular con el objetivo de poder reutilizar 
rutinas en el futuro, como para la factorización LU.

\subsection{Discusión}
La tarea de implementar el procediimento de Gauss-Jordan fue exitosa. 
Nuestros resultados de resolver los ejercicios de la parte 1 utilizando 
la implementación de Gauss-Jordan son consistentes con nuestros
resultados de los ejercicios hechos a mano de la parte 1. 
Considerando que los ejercicios de 
la parte 1 consisten en un varios tipos de sistemas lineales de ecuaciones,
como inhómogeneos, homogéneos con única e infinitas soluciones, y sistemas 
con número de ecuaciones distinto al número de variables; tenemos
la certeza de haber probado exitosamente 
nuestra implementación en varios tipos de sistemas de ecuaciones lineales. 

Hemos de resaltar que la motivación detrás de nuestras rutinas con tareas 
tan específicas para implementar Gauss-Jordan es para dejar abierta la posibilidad
a reutilizar las mismas rutinas para implementaciones en el futuro.

\end{document}