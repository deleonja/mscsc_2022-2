\documentclass[11pt,letterpaper]{article}
\usepackage[utf8]{inputenc}
\usepackage[spanish]{babel}
\usepackage{amsmath}
\usepackage{amsfonts}
\usepackage{amssymb}
\usepackage{physics}
\usepackage{hyperref}
\usepackage[left=2cm,right=2cm,top=2cm,bottom=2cm]{geometry}

\usepackage[draft,inline,nomargin]{fixme} \fxsetup{theme=color}
\definecolor{jacolor}{RGB}{200,40,0} \FXRegisterAuthor{ja}{aja}{\color{jacolor}JA}

\newcommand{\mcM}{\mathcal{M}}

\renewcommand{\labelenumii}{\arabic{enumi}.\arabic{enumii}}
\renewcommand{\labelenumiii}{\arabic{enumi}.\arabic{enumii}.\arabic{enumiii}}
\renewcommand{\labelenumiv}{\arabic{enumi}.\arabic{enumii}.\arabic{enumiii}.\arabic{enumiv}}

%%%%% Author
\author{José Alfredo de León}

%%%% Title 
\title{Tarea 3\\
\large{Métodos de Simulación Computacional para Sistemas Cuánticos - 2022-2}}


\begin{document}
\date{28 de febrero de 2022}
\maketitle

\section{Objetivos}
\subsection{Objetivo general}
Implementar mínimos cuadrados.

\subsection{Objetivos específicos}
\begin{enumerate}
\item 
\end{enumerate}

\section{Marco teórico}
El problema de ajustar una función a datos experimentales. Uno ajusta una
función y luego cómo prueba que la hipótesis de que esa función le ajusta
es cierta? Prueba de chi cuadrado.


\section{Parte 1: Ejercicios a mano}


\section{Parte 2: Implementación Gauss-Jordan}


\subsection{Implementación}


\subsection{Discusión}


\end{document}